\section{模型结果}

\NewDocumentCommand\MyCompare{m m m}{($#1$\ vs.\ $#2$, $p<#3$)}

\subsection{基准特征}

从eICU数据库中共提取出$100,308$条数据,包含$17,729$名不同的脓毒症患者。%
其中,$3,866\,(3.85\%)$条数据为正例,$96,442\,(96.15\%)$条数据为反例。

经过比较,正例拥有更长的ICU入住天数\MyCompare{21.067}{10.852}{0.001},%
更少的血浆蛋白\MyCompare{2.109}{2.520}{0.001},%
更少的淋巴细胞数目\MyCompare{9.931}{12.473}{0.001},%
更高的心率\MyCompare{93.337}{88.458}{0.001},%
更高的呼吸频率\MyCompare{21.814}{21.019}{0.001},%
更少的血清总蛋白\MyCompare{5.578}{5.928}{0.001},%
更低的红细胞比容\MyCompare{27.808}{29.888}{0.001},%
更少的肌酸酐\MyCompare{1.489}{1.610}{0.001},%
更高的白细胞计数\MyCompare{13.218}{12.189}{0.001},%
更多的血小板\MyCompare{260.259}{226.342}{0.001},%
更低的平均动脉压\MyCompare{79.727}{82.055}{0.001}。

\subsection{模型比较}

\begin{table}[htbp]
    \centering
    \scriptsize
    \begin{threeparttable}
        \begin{tabular}{clcc}
            \toprule
            排名 & 模型名称                         & 平均准确率               & 平均AUC\tnote{1}      \\
            \midrule
            1  & CatBoost                     & $0.996 (\pm 0.001)$ & $0.996 (\pm 0.001)$ \\
            2  & Light Gradient Boosting      & $0.995 (\pm 0.001)$ & $0.996 (\pm 0.001)$ \\
            3  & Extreme Gradient Boosting    & $0.995 (\pm 0.001)$ & $0.994 (\pm 0.002)$ \\
            4  & Hist Gradient Boosting       & $0.994 (\pm 0.002)$ & $0.996 (\pm 0.002)$ \\
            5  & Ada Boost                    & $0.993 (\pm 0.002)$ & $0.995 (\pm 0.002)$ \\
            6  & Decision Tree                & $0.989 (\pm 0.002)$ & $0.949 (\pm 0.013)$ \\
            7  & Multi-Layer Perceptron       & $0.982 (\pm 0.004)$ & $0.975 (\pm 0.008)$ \\
            8  & SVM (RBF Kernel)             & $0.973 (\pm 0.003)$ & $0.957 (\pm 0.011)$ \\
            9  & Logistic                     & $0.966 (\pm 0.007)$ & $0.956 (\pm 0.012)$ \\
            10 & Extra Trees                  & $0.961 (\pm 0.006)$ & $0.977 (\pm 0.006)$ \\
            11 & Naive Bayes                  & $0.961 (\pm 0.006)$ & $0.689 (\pm 0.034)$ \\
            12 & Ridge                        & $0.961 (\pm 0.007)$ & $0.952 (\pm 0.013)$ \\
            13 & Linear Discriminant Analysis & $0.961 (\pm 0.010)$ & $0.952 (\pm 0.013)$ \\
            14 & K-Nearest Neighbours         & $0.951 (\pm 0.006)$ & $0.544 (\pm 0.025)$ \\
            \bottomrule
        \end{tabular}
        \begin{tablenotes}
            \tiny
            \item[1] AUC:Area Under Curve,接受者操作特性曲线下与坐标轴围成的面积。
        \end{tablenotes}
    \end{threeparttable}
    \caption{$14$种模型的交叉验证结果比较(按平均准确率排序)}
    \label{table:model-comparison}
\end{table}

用提取出的数据训练预测模型,各种模型的交叉验证结果如表\ref{table:model-comparison}所示。%
Logistic回归表现良好(平均准确率:$0.966$,平均AUC:$0.956$),%
而集成学习方法拥有更高的平均准确率和平均AUC。%
其中,CatBoost的预测结果最好(平均准确率:$0.996$,平均AUC:$0.996$),%
故选择CatBoost进入下一步。

\subsection{完整模型与紧凑模型}

\begin{figure}[htbp]
    \centering
    \includegraphics[width=0.9\linewidth]{../img/eicu_full_shap_beeswarm_20.png}
    \caption{完整模型中各变量的平均SHAP值比较}
    \label{figure:full-shap}
\end{figure}

根据预测结果比较,选择含$57$个输入变量的CatBoost模型为完整模型。%
计算完整模型中各变量的平均SHAP值,结果如图\ref{figure:full-shap}所示。%
此摘要图展示了各个变量对预测结果的影响情况分布。%
例如,ICU入住天数(offset)对结果影响明显,且ICU入住天数越长,发生ICU综合症的概率越大。

TODO:

\subsection{性能分析}

TODO:

\subsection{模型解释}

TODO:

\subsection{H5预测工具}

TODO:
